%=== CHAPTER SIX (6) ===
%=== Decision Rules ===

\chapter{Decision Rules: When Can Cheap Methods Substitute for DIIM?}

The previous chapter showed that cheap methods can match DIIM at specific $q_0$ points but not universally. This chapter develops a systematic framework to determine \emph{when} each cheap method can be trusted, using Monte Carlo simulation and logistic regression.

\section{Framework}

\subsection{Motivation}

All cheap methods produce a \emph{fixed} sector ranking---they depend only on the IO table structure ($A$, $x$), not on the disruption profile ($q_0$). The DIIM, by contrast, adapts its ranking to the specific $q_0$. A cheap method performs well when its fixed ranking happens to align with where $q_0$ concentrates; it fails when the disruption hits sectors that the cheap method did not prioritise.

This insight suggests that the effectiveness of a cheap method is predictable from features of $q_0$. If we can identify these features, we can construct decision rules: ``Use cheap method $m$ if [condition on $q_0$]; otherwise, run the full DIIM.''

\subsection{Performance Metric}

For each cheap method~$m$ and a given $q_0$, we compute the performance ratio from Equation~\eqref{eq:performance-ratio}. Method~$m$ is considered ``close enough'' if $R_m \geq 0.95$.


\section{Monte Carlo Simulation}

We draw 2\,000 random $q_0$ vectors for each scenario (COVID-19, 15~sectors; Manpower, 107~sectors). For each trial:

\begin{enumerate}
    \item Draw $q_{0,i} \sim \text{Uniform}(10^{-6},\; 2 \max(q_0^{\text{actual}}))$ independently for each sector $i$.
    \item Run the DIIM baseline (no intervention) and compute total economic loss.
    \item Run the DIIM with intervention in the DIIM's top-5 sectors (10\% $q_0$ reduction) and compute $\Delta EL_{\text{DIIM}}$.
    \item For each cheap method~$m$, run the DIIM with intervention in method~$m$'s top-5 sectors and compute $\Delta EL_m$.
    \item Compute $R_m = \Delta EL_m / \Delta EL_{\text{DIIM}}$ and record whether $R_m \geq 0.95$ (``close'').
\end{enumerate}


\section{COVID-19 Results}

Table~\ref{tab:covid-decision-results} summarises the Monte Carlo results for the COVID-19 scenario.

\begin{table}[H]
\centering
\begin{tabular}{@{}lccc@{}}
\toprule
\textbf{Method} & \textbf{Close Rate ($\geq$95\%)} & \textbf{Win Rate} & \textbf{Mean Ratio} \\
\midrule
Output-Weighted Linkage & 88.9\% & 80.3\% & 1.156 \\
PCA                     & 85.5\% & 75.2\% & 1.095 \\
Forward Linkage         & 56.1\% & 44.2\% & 0.945 \\
Network Centrality      & 51.4\% & 40.1\% & 0.986 \\
Backward Linkage        &  2.3\% &  1.6\% & 0.478 \\
\bottomrule
\end{tabular}
\caption{Monte Carlo results for COVID-19 (2\,000 trials, $k=5$).}
\label{tab:covid-decision-results}
\end{table}

Output-weighted linkage and PCA both achieve close rates above 85\%, with mean performance ratios exceeding~1 (i.e., they \emph{outperform} DIIM on average). This occurs because these methods capture structural importance that sometimes identifies sectors the DIIM's loss-based ranking misses.

\begin{figure}[tbp]
    \centering
    \adjustbox{max width=\linewidth}{\includegraphics{Chapter6/covid_close_rates.png}}
    \caption{Close rates and win rates for each cheap method (COVID-19).}
    \label{fig:covid-close-rates}
\end{figure}

Figure~\ref{fig:covid-ratio-dist} shows the distribution of performance ratios across the 2\,000 trials. Output-weighted linkage and PCA have distributions centred above~1, while backward linkage is centred below~0.5.

\begin{figure}[tbp]
    \centering
    \adjustbox{max width=\linewidth}{\includegraphics{Chapter6/covid_ratio_dist.png}}
    \caption{Performance ratio distributions (COVID-19).}
    \label{fig:covid-ratio-dist}
\end{figure}


\section{Manpower Results}

Table~\ref{tab:manpower-decision-results} summarises the manpower scenario results.

\begin{table}[H]
\centering
\begin{tabular}{@{}lccc@{}}
\toprule
\textbf{Method} & \textbf{Close Rate ($\geq$95\%)} & \textbf{Win Rate} & \textbf{Mean Ratio} \\
\midrule
Output-Weighted Linkage & 55.4\% & 24.3\% & 0.942 \\
Forward Linkage         &  8.6\% &  0.9\% & 0.822 \\
PCA                     &  5.3\% &  0.4\% & 0.802 \\
Network Centrality      &  1.1\% &  0.0\% & 0.641 \\
Backward Linkage        &  0.0\% &  0.0\% & 0.175 \\
\bottomrule
\end{tabular}
\caption{Monte Carlo results for Manpower (2\,000 trials, $k=5$).}
\label{tab:manpower-decision-results}
\end{table}

The manpower scenario is much harder for all cheap methods. PCA's close rate drops from 85.5\% (COVID-19) to 5.3\% (manpower). Even output-weighted linkage, the best performer, only matches DIIM 55\% of the time. This is fundamentally driven by the number of sectors: with 107~sectors, the top-5 represent only 4.7\% of the economy, and the probability that any fixed ranking captures the five most impactful sectors under a random $q_0$ is much lower.

\begin{figure}[tbp]
    \centering
    \adjustbox{max width=\linewidth}{\includegraphics{Chapter6/manpower_close_rates.png}}
    \caption{Close rates and win rates for each cheap method (Manpower).}
    \label{fig:manpower-close-rates}
\end{figure}

\begin{figure}[tbp]
    \centering
    \adjustbox{max width=\linewidth}{\includegraphics{Chapter6/manpower_ratio_dist.png}}
    \caption{Performance ratio distributions (Manpower).}
    \label{fig:manpower-ratio-dist}
\end{figure}


\section{Decision Rules via Logistic Regression}

For each method~$m$, we fit a logistic regression model:

\begin{equation}
    \log \frac{P(R_m \geq 0.95)}{1 - P(R_m \geq 0.95)} = \beta_0 + \beta_1 f_1 + \beta_2 f_2 + \cdots
    \label{eq:logistic-regression}
\end{equation}

where the features $f_1, f_2, \ldots$ are derived from $q_0$ and the method's sector rankings:

\begin{itemize}
    \item \textbf{Method share}: the fraction of total $q_0$ captured by the method's top-5 sectors.
    \item \textbf{Method average rank}: the mean rank of the method's top-5 sectors in the $q_0$ ranking (normalised to [0,1]).
    \item \textbf{Overlap}: the number of the method's top-5 sectors that also appear in $q_0$'s top-5.
    \item \textbf{General $q_0$ features}: Gini coefficient, coefficient of variation, max-to-mean ratio, and normalised entropy of $q_0$.
\end{itemize}

For each model, we identify the most significant predictor and compute the Youden's~$J$ optimal threshold---the threshold that maximises sensitivity plus specificity minus~1.

Table~\ref{tab:decision-rules-summary} presents the derived decision rules.

\begin{table}[H]
\centering
\small
\begin{tabular}{@{}llcc@{}}
\toprule
\textbf{Scenario} & \textbf{Method} & \textbf{Decision Rule} & \textbf{$J$} \\
\midrule
\multirow{3}{*}{COVID-19}
 & Output-Weighted & OWL\_avgrank $< 0.533$ & 0.411 \\
 & PCA             & PCA\_avgrank $< 0.578$ & 0.304 \\
 & Network Cent.   & NC\_overlap $> 0.200$  & 0.197 \\
\midrule
\multirow{3}{*}{Manpower}
 & PCA             & PCA\_share $> 0.057$   & 0.559 \\
 & Forward Linkage & FL\_avgrank $< 0.456$  & 0.503 \\
 & Output-Weighted & OWL\_overlap $> 0$     & 0.102 \\
\bottomrule
\end{tabular}
\caption{Decision rules with Youden's $J$-optimal thresholds.}
\label{tab:decision-rules-summary}
\end{table}

The decision rules follow a common pattern: a cheap method works well when the disruption ($q_0$) is concentrated in sectors that the method has already identified as structurally important. The ``share'' and ``average rank'' features capture this alignment.


\section{Practical Recommendations}

Based on the analysis, we propose the following practical guidelines:

\begin{enumerate}
    \item \textbf{For small economies ($\leq$ 20 sectors)}: Output-weighted linkage or PCA can be used as primary screening tools. With close rates above 85\%, the risk of significant underperformance is low.
    \item \textbf{For large economies ($>$ 50 sectors)}: Output-weighted linkage should be used as a first-pass screen. If the overlap between its top-$k$ sectors and the observed high-$q_0$ sectors is low, the full DIIM should be run.
    \item \textbf{Pre-event planning}: Since $q_0$ is unknown before a disruption occurs, cheap methods are valuable for \emph{pre-positioning} resources. Output-weighted linkage provides the best structural estimate of which sectors are likely to be critical across a range of disruption profiles.
    \item \textbf{During a crisis}: Once $q_0$ is observed, the decision rules in Table~\ref{tab:decision-rules-summary} can be evaluated instantly to determine whether the cheap method's recommendation is trustworthy for the specific disruption at hand.
\end{enumerate}

%=== END OF CHAPTER SIX ===
\newpage
