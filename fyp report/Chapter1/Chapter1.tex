%=== CHAPTER ONE (1) ===
%=== INTRODUCTION ===

\chapter{Introduction}

When COVID-19 shut down the Singapore economy in April 2020, policymakers faced an urgent question: which sectors were suffering the most, and where should recovery resources be prioritised? While Gross domestic product (GDP) data showed that the economy contracted, they provided no insight into how disruptions rippled through interconnected sectors, what the recovery process was like, or how different policy interventions would trade off cost against benefit.

Given that  modern economies are highly interdependent, when one sector stops producing, it does not only affect workers in that sector but also deprives other sectors of the critical inputs they need. A disruption to the construction sector not only harms construction workers but also manufacturers of building materials, transport companies, and equipment suppliers. Understanding these cascading effects requires a framework that captures how economic sectors depend on each other and how disruptions propagate over time.

The Inoperability Input-Output Model (IIM) and its dynamic extension (DIIM) can provide this capability \cite{SantosHaimes2004,JinZhou2023}. The Shanghai COVID-19 study demonstrated how the DIIM could quantify which sectors experienced the highest production losses relative to its inital production level (inoperability) and which suffered the largest absolute economic damage \cite{JinZhou2023}. That research also showed how policymakers could use the model to compare alternative risk management strategies (eg, whether to invest in early prevention measures or in accelerating recovery) by calculating the return on investment for each option \cite{JinZhou2023}. This study applies the DIIM framework to Singapore's COVID-19 crisis to provide data for recovery policy decisions.


The Leontief input-output model, developed by 1973 Nobel laureate Wassily Leontief, is the foundation for understanding sectoral interdependencies \cite{NobelPrizeLeontiefFacts}. It represents how different sectors of an economy rely on each other's outputs as inputs, capturing the relationships that define an economy in equilibrium. The basic equation is simple: total output equals intermediate consumption plus final demand.

However, the Leontief model is static. It describes an economy at rest, not an economy in crisis and recovery. To analyse disruptions, researchers developed the Inoperability Input-Output Model (IIM), which measures "inoperability". This term is defined as the normalised production loss of each sector expressed on a scale from 0 (normal operation) to 1 (complete shutdown). The IIM measures how an initial shock in one sector propagates through input-output linkages to reduce production in other sectors. 

The DIIM extended this framework further by adding time. Real recoveries are not instantaneous. Sectors gradually restore production as supply-demand imbalances resolve. The DIIM models this recovery process using a resilience coefficient for each sector, showing how inoperability and economic losses evolve from the initial shock through full recovery. 

Singapore is a city-state heavily dependent on international trade. It experienced both the direct effects of lockdown restrictions and the indirect effects of global supply chain disruptions. With major economic sectors that interact intensively, from manufacturing and financial services to construction and hospitality, we believe that Singapore has been strongly affected by the pandemic. 

The objective of this study is to (1) quantify the impact of COVID-19 on Singapore's economy across sectors using the DIIM framework, (2) identify which sectors experienced the highest inoperability and economic losses, (3) understand how disruptions cascaded through supply chains, and (4) evaluate whether alternative risk management policies would have been cost-effective.





%=== END OF CHAPTER ONE ===
\newpage


