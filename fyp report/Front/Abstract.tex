%=== FRONT PART ===
%=== ABSTRACT ===

\chapter*{Abstract}
\addcontentsline{toc}{chapter}{Abstract}

This project develops a quantitative framework for assessing economic resilience under disruption using the Dynamic Inoperability Input-Output Model (DIIM) \cite{SantosHaimes2004,JinZhou2023}. The DIIM captures how sectoral disruptions propagate through input-output linkages and how the economy recovers over time. Two disruption scenarios relevant to Singapore are studied: (1)~the COVID-19 pandemic, using the 2019 input-output (IO) table aggregated to 15~sectors, and (2)~a manpower disruption scenario motivated by potential grey zone conflicts that could trigger large-scale foreign worker departures, using the 2022 IO table with 107~sectors.

A central challenge in disruption management is identifying which sectors should receive priority intervention. The DIIM can identify these ``key sectors'' by simulating the full disruption and ranking sectors by economic loss, but this approach is computationally expensive and requires crisis-time data (initial inoperability~$q_0$, demand perturbation~$c^*$) that may not be available during the early stages of an event. This motivates the search for cheaper structural alternatives that use only the IO table.

This study evaluates five cheap methods---Principal Component Analysis (PCA) on the integrated IO matrix, backward linkage, forward linkage, output-weighted linkage, and network centrality---against the DIIM benchmark through Monte Carlo simulation (2\,000 random $q_0$ vectors per scenario). Performance is measured by the ratio of economic loss reduction achieved by each cheap method relative to DIIM. For the 15-sector COVID scenario, output-weighted linkage matches DIIM in 89\% of cases, and PCA matches in 86\%. For the 107-sector manpower scenario, output-weighted linkage matches in 55\% of cases, while PCA drops to only 5\%. Logistic regression is used to derive decision rules that predict, from observable $q_0$ features, when each cheap method will perform adequately.

\par

%=== END OF ABSTRACT ===
