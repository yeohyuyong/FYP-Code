%=== FRONT PART ===
%=== ABSTRCT ===

%\begin{center}
\chapter*{Abstract}
%\end{center}
\addcontentsline{toc}{chapter}{Abstract}

This project studies the impact of COVID-19 on Singapore’s economy using the Dynamic Inoperability Input-Output Model (DIIM) \cite{SantosHaimes2004,JinZhou2023}. The DIIM is useful because it models how disruptions in one sector spread to other sectors through input-output linkages and how the economy recovers over time. Using Singapore’s 2019 input-output (IO) table \cite{DataGovSgIO2019}, the economy is aggregated into 15 sectors to match the workforce data used to estimate initial inoperability.

In this study, initial inoperability is calculated based on the share of unavailable workers in each sector due to Covid-19, adjusted by how dependent the sector is on labour as an input. The shock period is set to Singapore’s "circuit breaker" from 7 April to 1 June 2020 (55 days) \cite{MOH2020EndCircuitBreaker,MOH2020PostCircuitBreakerMeasures}, and recovery is simulated over a longer horizon, with a near-normal level assumed after 751 days. The model produces sector-level paths of inoperability and cumulative economic losses, which are used to identify vulnerable sectors and support recovery prioritisation.

A sensitivity analysis tests how changes in lockdown duration affect total losses. The project also evaluates example risk management policies using a surrogate worth trade-off (SWT) approach \cite{Haimes1975SWTChapter} to compare policy costs against the economic losses avoided. Finally, the study explores a PCA-based approach for identifying key sectors using only the IO table and compares its performance with DIIM-based selection through simulation.

\par
%\textbf{Keywords:} Dissertation, keywords.

%=== END OF CHAPTER ONE ===
