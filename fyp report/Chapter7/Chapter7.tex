%=== CHAPTER SEVEN (7) ===

\chapter{Conclusion}

This project applied the Dynamic Inoperability Input–Output Model (DIIM) to study how COVID-19 affected Singapore’s economy at the sector level, including how disruptions spread across industries and how recovery evolves over time. Using Singapore’s 2019 IO table aggregated to 15 sectors and an initial inoperability measure based on workforce unavailability and labour dependence, the model produced sectoral inoperability paths and cumulative economic loss estimates.

The results show that sector vulnerability should be assessed using both inoperability and economic loss because these measures capture different aspects of impact and can rank sectors differently. The impact matrix (ranking inoperability against ranking economic loss) provides a practical way to identify sectors that are consistently high-impact and should be prioritised when resources are limited. Sensitivity analysis suggests that longer lockdown durations increase losses across the economy, but the main set of highly affected sectors remains broadly consistent.

The dissertation also demonstrated how DIIM outputs can be used to compare example risk management policies using a surrogate worth trade-off (SWT) approach that accounts for both cost and benefit. Finally, a PCA-based method using the integrated IO matrix was tested as a faster alternative for identifying key sectors, and simulations suggest that it can match DIIM-based prioritisation in reducing losses while requiring fewer uncertain crisis-time inputs.

%=== END OF CHAPTER SEVEN ===
\newpage
