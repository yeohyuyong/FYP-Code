%=== CHAPTER FOUR (4) ===
%=== Case Study 2: Manpower Disruption ===

\chapter{Case Study 2: Manpower Disruption}

This chapter extends the DIIM framework to a manpower disruption scenario motivated by Singapore's dependence on foreign labour and the potential for grey zone conflicts to trigger workforce departures.

\section{Background: Singapore's Foreign Worker Dependence}

Singapore's economy relies on approximately 1.4~million foreign workers, who constitute roughly one-third of the total labour force \cite{MOM2023ForeignWorkforce}. These workers are concentrated in sectors such as construction, manufacturing, marine shipyard, process industries, and domestic services. While this reliance on foreign labour has been a key enabler of economic growth, it creates a structural vulnerability to events that could trigger rapid workforce departures.

Grey zone conflicts---coercive actions that fall below the threshold of conventional warfare---represent a plausible threat scenario \cite{MINDEF2020Factsheet}. Such conflicts could include economic sanctions, cyber attacks on critical infrastructure, disinformation campaigns, or military posturing in the region. Any of these could erode confidence among foreign workers and their home governments, potentially triggering a significant and rapid outflow of non-resident workers.

Unlike the COVID-19 scenario, where disruption arose from both workforce unavailability ($q_0 > 0$) and reduced demand ($c^* > 0$), a manpower disruption is characterised by:

\begin{itemize}
    \item \textbf{High initial inoperability} ($q_0 > 0$): Sectors lose productive capacity as workers depart.
    \item \textbf{No external demand shock} ($c^* = 0$): The disruption is purely structural. Demand may persist but cannot be met due to workforce shortages.
\end{itemize}

This distinction is important because the DIIM solution simplifies to $q(t) = e^{-K(I-A^*)t}\, q(0)$ when $c^* = 0$ (Equation~\eqref{eq:diim-solution-homogeneous}), meaning the disruption dynamics are governed entirely by the structure of the economy ($A^*$) and the initial shock ($q_0$).


\section{Data and Model Parameters}

\subsection{Input-Output Table}

This scenario uses the 2022 Singapore IO table, which provides a more granular sectoral breakdown of 107~sectors. The finer resolution is essential for capturing the heterogeneous distribution of foreign workers across sub-sectors.

\subsection{Initial Inoperability}

Initial inoperability is computed using the same method as Equation~\eqref{eq:initial-inoperability}, with workforce unavailability data reflecting the proportion of foreign workers in each sector. Sectors with high foreign worker concentrations (e.g., construction, process industries) have correspondingly higher initial inoperability.

\subsection{Model Parameters}

The shock duration is set to 55~days to enable comparison with the COVID-19 scenario, though a manpower disruption could have a different timeline. The total simulation horizon is 751~days with a final inoperability of 1\%, representing near-full recovery through workforce replacement, retraining, or automation.


\section{Results}

\subsection{Inoperability Evolution}

Figure~\ref{fig:manpower-inoperability} shows the simulated inoperability trajectories for the 107~sectors. Compared to the COVID-19 scenario, the manpower disruption exhibits:

\begin{itemize}
    \item Greater dispersion in initial inoperability across sectors, reflecting the uneven distribution of foreign workers.
    \item Stronger cascading effects through the 107-sector production network, as disruptions in labour-intensive sectors propagate to downstream industries.
\end{itemize}

\begin{figure}[tbp]
    \centering
    \adjustbox{max width=\linewidth}{\includegraphics{Chapter4/manpower_inoperability.png}}
    \caption{Inoperability evolution for the manpower disruption scenario (107 sectors).}
    \label{fig:manpower-inoperability}
\end{figure}


\subsection{Economic Loss Evolution}

Figure~\ref{fig:manpower-economic-loss} presents the cumulative economic loss evolution. The 107-sector resolution reveals that even sectors with moderate inoperability can accumulate substantial losses if their baseline output is large (e.g., manufacturing sub-sectors, financial services).

\begin{figure}[tbp]
    \centering
    \adjustbox{max width=\linewidth}{\includegraphics{Chapter4/manpower_econ_loss.png}}
    \caption{Economic loss evolution for the manpower disruption scenario (107 sectors).}
    \label{fig:manpower-economic-loss}
\end{figure}


\section{Comparison with COVID-19}

Table~\ref{tab:scenario-comparison} summarises the key structural differences between the two scenarios.

\begin{table}[tbp]
\centering
\begin{tabular}{@{}lcc@{}}
\toprule
\textbf{Parameter} & \textbf{COVID-19} & \textbf{Manpower} \\
\midrule
Number of sectors & 15 & 107 \\
IO table year & 2019 & 2022 \\
Demand shock $c^*$ & $>0$ & $= 0$ \\
Shock type & Demand + supply & Pure supply \\
Lockdown duration & 55 days & 55 days \\
Total horizon & 751 days & 751 days \\
\bottomrule
\end{tabular}
\caption{Comparison of COVID-19 and manpower disruption scenario parameters.}
\label{tab:scenario-comparison}
\end{table}

The manpower scenario is structurally more complex because (1)~the 107-sector economy has a denser and more heterogeneous interdependency matrix~$A^*$, and (2)~the absence of a demand shock means the entire disruption propagation is driven by the production network. These structural differences have important implications for the effectiveness of cheap key sector identification methods, as explored in the following chapters.

%=== END OF CHAPTER FOUR ===
\newpage
