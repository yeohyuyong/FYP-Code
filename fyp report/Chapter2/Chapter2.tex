%=== CHAPTER TWO (2) ===
%=== Literature Review ===

\chapter{Methodology}

This paper uses the DIIM model to assess the impacts of COVID-19 on the inoperability and economic losses of Singapore’s sectors. The DIIM model is derived from the Leontief input-output model and is used to investigate the higher-order transmission effect of input-output linkages between sectors on inoperability. Using the DIIM model requires first introducing the concept of “inoperability”, which is defined as follows:

% Requires: \usepackage{amsmath}
\begin{equation}
    \label{eq:inoperability-definition}
    \text{Inoperability} = \frac{\text{As Planned Production} - \text{Degraded Production}}{\text{As Planned Production}}
\end{equation}

Where "As Planned Production" represents the output level of a sector under normal production, and "Degraded Production" represents the output level of a sector after being shocked in its production process. In this paper, the shock comes from the reduction in demand due to the lockdown caused by COVID-19. The difference between "As Planned Production" and "Degraded Production" describes the degree of output decline of a sector. "Inoperability" takes values between 0 and 1, with higher values indicating greater damage to production caused by the shock. A value of 1 means the shocked sector has completely lost its production capacity, while a value of 0 means the sector is producing at a normal level.

Using the concept of inoperability defined above, the IIM model can be derived from the Leontief input-output model, as shown in Equation~\eqref{eq:leontief-model}.

\begin{equation}
    x = A x + c
    \label{eq:leontief-model}
\end{equation}

Where x is the total output vector; $A$ is the technical coefficient matrix; $c$ is the final demand vector. If we define the output levels and the final demand vector of some shocked sectors as $\tilde{x}$ and $\tilde{c}$ respectively, then based on equation (edit), we can construct the input-output relationship between sectors after the shock:

Subtracting Equation~\eqref{eq:leontief-model} from the shocked-economy formulation:

% Requires: \usepackage{amsmath}
\begin{equation}
    x - \tilde{x} = A (x - \tilde{x}) + (c - \tilde{c})
    \label{eq:leontief-difference}
\end{equation}

Defining $\hat{x}$ as a diagonalized matrix of the output vector and left multiplying $\hat{x}^{-1}$ on both sides of Equation~\eqref{eq:leontief-difference}:

% Requires: \usepackage{amsmath}
\begin{equation}
    \hat{x}^{-1}(x - \tilde{x}) = \hat{x}^{-1}A(x - \tilde{x}) + \hat{x}^{-1}(c - \tilde{c})
    \label{eq:normalized-difference}
\end{equation}

Let $q = \hat{x}^{-1}\left(x - \tilde{x}\right), \quad A^{*} = \hat{x}^{-1} A \hat{x}, \quad c^{*} = \hat{x}^{-1}\left(c - \tilde{c}\right)$; then Equation~\eqref{eq:iim-model} can be derived, which is the IIM model.

% Requires: \usepackage{amsmath}
\begin{equation}
    q = A^{*} q + c^{*}
    \label{eq:iim-model}
\end{equation}

The details of each matrix in the IIM model are shown in Equations~\eqref{eq:q-definition}, \eqref{eq:astar-definition}, and~\eqref{eq:cstar-definition}:

% Requires: \usepackage{amsmath}
\begin{equation}
    \label{eq:q-definition}
    q = \hat{x}^{-1} (x - \tilde{x}) =
    \begin{bmatrix}
        \dfrac{1}{x_1} & 0 & \cdots & \cdots & 0 \\
        0 & \ddots & & & \vdots \\
        \vdots & & \dfrac{1}{x_i} & & \vdots \\
        \vdots & & & \ddots & 0 \\
        0 & \cdots & \cdots & 0 & \dfrac{1}{x_n}
    \end{bmatrix}
    \begin{bmatrix}
        x_1 - \tilde{x}_1 \\
        \vdots \\
        x_i - \tilde{x}_i \\
        \vdots \\
        x_n - \tilde{x}_n
    \end{bmatrix}
\end{equation}

% Requires: \usepackage{amsmath}
\begin{equation}
    A^{*} = \hat{x}^{-1} A \hat{x} =
    \begin{bmatrix}
        a_{11}\dfrac{x_{1}}{x_{1}} & \cdots & a_{1j}\dfrac{x_{j}}{x_{1}} & \cdots & a_{1n}\dfrac{x_{n}}{x_{1}} \\
        \vdots & \ddots & \vdots & \ddots & \vdots \\
        a_{i1}\dfrac{x_{1}}{x_{i}} & \cdots & a_{ij}\dfrac{x_{j}}{x_{i}} & \cdots & a_{in}\dfrac{x_{n}}{x_{i}} \\
        \vdots & \ddots & \vdots & \ddots & \vdots \\
        a_{n1}\dfrac{x_{1}}{x_{n}} & \cdots & a_{nj}\dfrac{x_{j}}{x_{n}} & \cdots & a_{nn}\dfrac{x_{n}}{x_{n}}
    \end{bmatrix}
    \label{eq:astar-definition}
\end{equation}


% Requires: \usepackage{amsmath}
\begin{equation}
    \label{eq:cstar-definition}
    \mathbf{c}^{*} = \hat{\mathbf{x}}^{-1} \left( \mathbf{c} - \tilde{\mathbf{c}} \right)
    =
    \begin{bmatrix}
        \dfrac{1}{x_{1}} & 0 & \cdots & 0 \\
        0 & \ddots & \ddots & \vdots \\
        \vdots & \ddots & \dfrac{1}{x_{i}} & 0 \\
        \vdots & & \ddots & \vdots \\
        0 & \cdots & \cdots & \dfrac{1}{x_{n}}
    \end{bmatrix}
    \begin{bmatrix}
        c_{1} - \tilde{c}_{1} \\
        \vdots \\
        c_{i} - \tilde{c}_{i} \\
        \vdots \\
        c_{n} - \tilde{c}_{n}
    \end{bmatrix}
\end{equation}

The IIM model can be extended to the DIIM model by introducing a time variable and an elasticity coefficient matrix describing the recovery capacity of sectors. The discrete form of the DIIM model is:

% Requires: \usepackage{amsmath}
\begin{equation}
    q(t+1) = q(t) + K \big[ A^{*} q(t) + c^{*}(t) - q(t) \big]
    \label{eq:diim-discrete}
\end{equation}

Where K is the elasticity coefficient matrix describing the recovery capacity of sectors after being shocked in production. Assuming that the recovery capacity of a sector depends only on its own production, and is unrelated to the production linkages with other sectors, thus K is a diagonal matrix with diagonal elements greater than 0. The larger the diagonal elements, the stronger the recovery capacity of the corresponding sector in response to shocks. t is the discrete time variable, and q(t) represents the sectoral inoperability vector at time t. Equation~\eqref{eq:diim-discrete} states that the inoperability of a sector in a period depends on the inoperability in the previous period and the sector’s recovery adjustment capacity. Approximating Equation~\eqref{eq:diim-discrete} to differential form:

% Requires: \usepackage{amsmath}
\begin{equation}
    \label{eq:diim-continuous}
    \dot{q}(t) = K \bigl[ A^{*} q(t) + c^{*}(t) - q(t) \bigr]
\end{equation}

Solving Equation~\eqref{eq:diim-continuous} we can get the equation describing the evolution of sectoral inoperability over time:

% Requires: \usepackage{amsmath}
\begin{equation}
    q(t) = e^{-K(I-A^{\ast})t} q(0) + \int_{0}^{t} K e^{-K(I-A^{\ast})(t-z)} c^{\ast}(z)\, dz
    \label{eq:diim-solution-general}
\end{equation}

Assuming that the demand shock $c^{*}$ remains unchanged, and $c^{*}=0$, then Equation~\eqref{eq:diim-solution-general} can be simplified to:

% Requires: \usepackage{amsmath}
\begin{equation}
    q(t) = e^{-K (I - A^{*}) t} q(0)
    \label{eq:diim-solution-homogeneous}
\end{equation}

Where $q(0)$ represents the initial inoperability vector of sectors after being shocked. As time goes by, the inoperability changes at a rate of $e^{-K(I - A^*)t}$. The initial inoperability $q(0)$ is calculated as:

% Requires: \usepackage{amsmath}
\begin{equation}
    \label{eq:initial-inoperability}
    \text{Sector Initial Inoperability} = \frac{\text{Unavailable Workforce}}{\text{Size of Workforce}} * \frac{\text{LAPI}}{\text{Sector Output}}
\end{equation}

where the left side of the formula is the ratio between the number of unavailable workers due to COVID-19 in each sector and the total number of workers in each sector, and the right hand side is how much each sector depends on its workforce as inputs.

Finally, we calculate the elasticity coefficient matrix $K$ describing the recovery capacity of sectors. According to Equation~\eqref{eq:diim-solution-homogeneous}, the inoperability of sector $i$ at time $t$ is:

% Requires: \usepackage{amsmath}
\begin{equation}
    q_i(t) = q_i(0) e^{-k_i \left( 1 - a_{ii}^\ast \right) t}
    \label{eq:qi-exponential-decay}
\end{equation}

From Equation~\eqref{eq:qi-exponential-decay}, we can derive the calculation formula for $k_i$:

% Requires: \usepackage{amsmath}
\begin{equation}
    k_i = \frac{\ln\left[q_i(0)/q_i(T)\right]}{T_i\left(1 - a_{ii}^*\right)}
    \label{eq:ki-definition}
\end{equation}

Where $q_i(0)$ is the initial inoperability of sector $i$; $q_i(T)$ is the inoperability of sector $i$ at time $T$; $a_{ii}^{*}$
is the element on the diagonal of matrix $A^*$, representing the self-dependence of sector $i$.

Under the analytical framework of the DIIM model, the economic losses caused by the shock to sector can be represented by Equation~\eqref{eq:economic-loss}:

% Requires: \usepackage{amsmath}
\begin{equation}
    EL_i = x_i \int_{t=0}^{t=T} q_i(t) \, dt
    \label{eq:economic-loss}
\end{equation}

Where $x_i$ is the output value of sector $i$ at time $t$ under normal production, and $n$ is the number of sectors used in the input-output table of this
study.


%=== END OF CHAPTER TWO ===
\newpage
