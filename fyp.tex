% Options for packages loaded elsewhere
\PassOptionsToPackage{unicode}{hyperref}
\PassOptionsToPackage{hyphens}{url}
\PassOptionsToPackage{dvipsnames,svgnames,x11names}{xcolor}
%
\documentclass[
  letterpaper,
  DIV=11,
  numbers=noendperiod]{scrartcl}

\usepackage{amsmath,amssymb}
\usepackage{iftex}
\ifPDFTeX
  \usepackage[T1]{fontenc}
  \usepackage[utf8]{inputenc}
  \usepackage{textcomp} % provide euro and other symbols
\else % if luatex or xetex
  \usepackage{unicode-math}
  \defaultfontfeatures{Scale=MatchLowercase}
  \defaultfontfeatures[\rmfamily]{Ligatures=TeX,Scale=1}
\fi
\usepackage{lmodern}
\ifPDFTeX\else  
    % xetex/luatex font selection
\fi
% Use upquote if available, for straight quotes in verbatim environments
\IfFileExists{upquote.sty}{\usepackage{upquote}}{}
\IfFileExists{microtype.sty}{% use microtype if available
  \usepackage[]{microtype}
  \UseMicrotypeSet[protrusion]{basicmath} % disable protrusion for tt fonts
}{}
\makeatletter
\@ifundefined{KOMAClassName}{% if non-KOMA class
  \IfFileExists{parskip.sty}{%
    \usepackage{parskip}
  }{% else
    \setlength{\parindent}{0pt}
    \setlength{\parskip}{6pt plus 2pt minus 1pt}}
}{% if KOMA class
  \KOMAoptions{parskip=half}}
\makeatother
\usepackage{xcolor}
\setlength{\emergencystretch}{3em} % prevent overfull lines
\setcounter{secnumdepth}{5}
% Make \paragraph and \subparagraph free-standing
\ifx\paragraph\undefined\else
  \let\oldparagraph\paragraph
  \renewcommand{\paragraph}[1]{\oldparagraph{#1}\mbox{}}
\fi
\ifx\subparagraph\undefined\else
  \let\oldsubparagraph\subparagraph
  \renewcommand{\subparagraph}[1]{\oldsubparagraph{#1}\mbox{}}
\fi

\usepackage{color}
\usepackage{fancyvrb}
\newcommand{\VerbBar}{|}
\newcommand{\VERB}{\Verb[commandchars=\\\{\}]}
\DefineVerbatimEnvironment{Highlighting}{Verbatim}{commandchars=\\\{\}}
% Add ',fontsize=\small' for more characters per line
\usepackage{framed}
\definecolor{shadecolor}{RGB}{241,243,245}
\newenvironment{Shaded}{\begin{snugshade}}{\end{snugshade}}
\newcommand{\AlertTok}[1]{\textcolor[rgb]{0.68,0.00,0.00}{#1}}
\newcommand{\AnnotationTok}[1]{\textcolor[rgb]{0.37,0.37,0.37}{#1}}
\newcommand{\AttributeTok}[1]{\textcolor[rgb]{0.40,0.45,0.13}{#1}}
\newcommand{\BaseNTok}[1]{\textcolor[rgb]{0.68,0.00,0.00}{#1}}
\newcommand{\BuiltInTok}[1]{\textcolor[rgb]{0.00,0.23,0.31}{#1}}
\newcommand{\CharTok}[1]{\textcolor[rgb]{0.13,0.47,0.30}{#1}}
\newcommand{\CommentTok}[1]{\textcolor[rgb]{0.37,0.37,0.37}{#1}}
\newcommand{\CommentVarTok}[1]{\textcolor[rgb]{0.37,0.37,0.37}{\textit{#1}}}
\newcommand{\ConstantTok}[1]{\textcolor[rgb]{0.56,0.35,0.01}{#1}}
\newcommand{\ControlFlowTok}[1]{\textcolor[rgb]{0.00,0.23,0.31}{#1}}
\newcommand{\DataTypeTok}[1]{\textcolor[rgb]{0.68,0.00,0.00}{#1}}
\newcommand{\DecValTok}[1]{\textcolor[rgb]{0.68,0.00,0.00}{#1}}
\newcommand{\DocumentationTok}[1]{\textcolor[rgb]{0.37,0.37,0.37}{\textit{#1}}}
\newcommand{\ErrorTok}[1]{\textcolor[rgb]{0.68,0.00,0.00}{#1}}
\newcommand{\ExtensionTok}[1]{\textcolor[rgb]{0.00,0.23,0.31}{#1}}
\newcommand{\FloatTok}[1]{\textcolor[rgb]{0.68,0.00,0.00}{#1}}
\newcommand{\FunctionTok}[1]{\textcolor[rgb]{0.28,0.35,0.67}{#1}}
\newcommand{\ImportTok}[1]{\textcolor[rgb]{0.00,0.46,0.62}{#1}}
\newcommand{\InformationTok}[1]{\textcolor[rgb]{0.37,0.37,0.37}{#1}}
\newcommand{\KeywordTok}[1]{\textcolor[rgb]{0.00,0.23,0.31}{#1}}
\newcommand{\NormalTok}[1]{\textcolor[rgb]{0.00,0.23,0.31}{#1}}
\newcommand{\OperatorTok}[1]{\textcolor[rgb]{0.37,0.37,0.37}{#1}}
\newcommand{\OtherTok}[1]{\textcolor[rgb]{0.00,0.23,0.31}{#1}}
\newcommand{\PreprocessorTok}[1]{\textcolor[rgb]{0.68,0.00,0.00}{#1}}
\newcommand{\RegionMarkerTok}[1]{\textcolor[rgb]{0.00,0.23,0.31}{#1}}
\newcommand{\SpecialCharTok}[1]{\textcolor[rgb]{0.37,0.37,0.37}{#1}}
\newcommand{\SpecialStringTok}[1]{\textcolor[rgb]{0.13,0.47,0.30}{#1}}
\newcommand{\StringTok}[1]{\textcolor[rgb]{0.13,0.47,0.30}{#1}}
\newcommand{\VariableTok}[1]{\textcolor[rgb]{0.07,0.07,0.07}{#1}}
\newcommand{\VerbatimStringTok}[1]{\textcolor[rgb]{0.13,0.47,0.30}{#1}}
\newcommand{\WarningTok}[1]{\textcolor[rgb]{0.37,0.37,0.37}{\textit{#1}}}

\providecommand{\tightlist}{%
  \setlength{\itemsep}{0pt}\setlength{\parskip}{0pt}}\usepackage{longtable,booktabs,array}
\usepackage{calc} % for calculating minipage widths
% Correct order of tables after \paragraph or \subparagraph
\usepackage{etoolbox}
\makeatletter
\patchcmd\longtable{\par}{\if@noskipsec\mbox{}\fi\par}{}{}
\makeatother
% Allow footnotes in longtable head/foot
\IfFileExists{footnotehyper.sty}{\usepackage{footnotehyper}}{\usepackage{footnote}}
\makesavenoteenv{longtable}
\usepackage{graphicx}
\makeatletter
\def\maxwidth{\ifdim\Gin@nat@width>\linewidth\linewidth\else\Gin@nat@width\fi}
\def\maxheight{\ifdim\Gin@nat@height>\textheight\textheight\else\Gin@nat@height\fi}
\makeatother
% Scale images if necessary, so that they will not overflow the page
% margins by default, and it is still possible to overwrite the defaults
% using explicit options in \includegraphics[width, height, ...]{}
\setkeys{Gin}{width=\maxwidth,height=\maxheight,keepaspectratio}
% Set default figure placement to htbp
\makeatletter
\def\fps@figure{htbp}
\makeatother

\KOMAoption{captions}{tableheading}
\makeatletter
\@ifpackageloaded{caption}{}{\usepackage{caption}}
\AtBeginDocument{%
\ifdefined\contentsname
  \renewcommand*\contentsname{Table of contents}
\else
  \newcommand\contentsname{Table of contents}
\fi
\ifdefined\listfigurename
  \renewcommand*\listfigurename{List of Figures}
\else
  \newcommand\listfigurename{List of Figures}
\fi
\ifdefined\listtablename
  \renewcommand*\listtablename{List of Tables}
\else
  \newcommand\listtablename{List of Tables}
\fi
\ifdefined\figurename
  \renewcommand*\figurename{Figure}
\else
  \newcommand\figurename{Figure}
\fi
\ifdefined\tablename
  \renewcommand*\tablename{Table}
\else
  \newcommand\tablename{Table}
\fi
}
\@ifpackageloaded{float}{}{\usepackage{float}}
\floatstyle{ruled}
\@ifundefined{c@chapter}{\newfloat{codelisting}{h}{lop}}{\newfloat{codelisting}{h}{lop}[chapter]}
\floatname{codelisting}{Listing}
\newcommand*\listoflistings{\listof{codelisting}{List of Listings}}
\makeatother
\makeatletter
\makeatother
\makeatletter
\@ifpackageloaded{caption}{}{\usepackage{caption}}
\@ifpackageloaded{subcaption}{}{\usepackage{subcaption}}
\makeatother
\ifLuaTeX
  \usepackage{selnolig}  % disable illegal ligatures
\fi
\usepackage{bookmark}

\IfFileExists{xurl.sty}{\usepackage{xurl}}{} % add URL line breaks if available
\urlstyle{same} % disable monospaced font for URLs
\hypersetup{
  pdftitle={DIIM},
  colorlinks=true,
  linkcolor={blue},
  filecolor={Maroon},
  citecolor={Blue},
  urlcolor={Blue},
  pdfcreator={LaTeX via pandoc}}

\title{DIIM}
\author{}
\date{}

\begin{document}
\maketitle

\renewcommand*\contentsname{Table of contents}
{
\hypersetup{linkcolor=}
\setcounter{tocdepth}{3}
\tableofcontents
}
\subsubsection{Download libraries}\label{download-libraries}

\begin{Shaded}
\begin{Highlighting}[]
\FunctionTok{library}\NormalTok{(openxlsx)}
\FunctionTok{library}\NormalTok{(ggplot2)}
\FunctionTok{library}\NormalTok{(reshape2)}
\FunctionTok{library}\NormalTok{(gridExtra)}
\FunctionTok{library}\NormalTok{(dplyr)}
\FunctionTok{library}\NormalTok{(tidyverse)}
\FunctionTok{source}\NormalTok{(}\StringTok{"functions.R"}\NormalTok{)}
\end{Highlighting}
\end{Shaded}

\begin{Shaded}
\begin{Highlighting}[]
\NormalTok{data }\OtherTok{=} \FunctionTok{download\_data}\NormalTok{()}
\NormalTok{A }\OtherTok{=}\NormalTok{ data}\SpecialCharTok{$}\NormalTok{A}
\NormalTok{x }\OtherTok{=}\NormalTok{ data}\SpecialCharTok{$}\NormalTok{x}
\NormalTok{c }\OtherTok{=}\NormalTok{ data}\SpecialCharTok{$}\NormalTok{c}
\NormalTok{q0 }\OtherTok{=}\NormalTok{ data}\SpecialCharTok{$}\NormalTok{q0}
\NormalTok{c\_star }\OtherTok{=}\NormalTok{ data}\SpecialCharTok{$}\NormalTok{c\_star}
\NormalTok{A\_star }\OtherTok{=}\NormalTok{ data}\SpecialCharTok{$}\NormalTok{A\_star}

  
\NormalTok{DIIM\_model }\OtherTok{=} \FunctionTok{DIIM}\NormalTok{(q0,A\_star,c\_star,x,}\AttributeTok{lockdown\_duration=}\DecValTok{55}\NormalTok{, }\AttributeTok{total\_duration=}\DecValTok{751}\NormalTok{)}
\end{Highlighting}
\end{Shaded}

\subsection{Inoperability Plot}\label{inoperability-plot}

\begin{Shaded}
\begin{Highlighting}[]
\CommentTok{\# Calculate maximum inoperability for each sector over time}
\NormalTok{inoperability\_evolution }\OtherTok{=}\NormalTok{ DIIM\_model}\SpecialCharTok{$}\NormalTok{inoperability\_evolution}
\NormalTok{max\_inoperability }\OtherTok{\textless{}{-}} \FunctionTok{apply}\NormalTok{(inoperability\_evolution, }\DecValTok{1}\NormalTok{, max)}

\CommentTok{\# Get the order of sectors sorted by descending maximum inoperability}
\NormalTok{sorted\_indices }\OtherTok{\textless{}{-}} \FunctionTok{order}\NormalTok{(max\_inoperability, }\AttributeTok{decreasing =} \ConstantTok{TRUE}\NormalTok{)}

\CommentTok{\# Reorder the inoperability evolution matrix}
\NormalTok{sorted\_inoperability }\OtherTok{\textless{}{-}}\NormalTok{ inoperability\_evolution[sorted\_indices, ]}

\CommentTok{\# Optional: get sector names if available (replace \textquotesingle{}sector\_names\textquotesingle{})}
\CommentTok{\# sector\_names\_sorted \textless{}{-} sector\_names[sorted\_indices]}

\CommentTok{\# Setup colors and labels for plotting}
\NormalTok{num\_sectors }\OtherTok{\textless{}{-}} \FunctionTok{nrow}\NormalTok{(sorted\_inoperability)}
\FunctionTok{matplot}\NormalTok{(}\FunctionTok{t}\NormalTok{(sorted\_inoperability), }\AttributeTok{type =} \StringTok{\textquotesingle{}l\textquotesingle{}}\NormalTok{, }\AttributeTok{lty =} \DecValTok{1}\NormalTok{, }\AttributeTok{col =} \FunctionTok{rainbow}\NormalTok{(num\_sectors),}
        \AttributeTok{xlab =} \StringTok{\textquotesingle{}Days\textquotesingle{}}\NormalTok{, }\AttributeTok{ylab =} \StringTok{\textquotesingle{}Inoperability\textquotesingle{}}\NormalTok{,}
        \AttributeTok{main =} \StringTok{\textquotesingle{}Inoperability Evolution\textquotesingle{}}\NormalTok{)}
\FunctionTok{legend}\NormalTok{(}\StringTok{\textquotesingle{}topright\textquotesingle{}}\NormalTok{, }\AttributeTok{legend =} \FunctionTok{paste}\NormalTok{(}\StringTok{\textquotesingle{}Sector\textquotesingle{}}\NormalTok{, sorted\_indices), }\AttributeTok{col =} \FunctionTok{rainbow}\NormalTok{(num\_sectors), }\AttributeTok{lty =} \DecValTok{1}\NormalTok{, }\AttributeTok{cex =} \FloatTok{0.6}\NormalTok{)}
\end{Highlighting}
\end{Shaded}

\includegraphics{fyp_files/figure-pdf/unnamed-chunk-3-1.pdf}

\subsection{Economic Loss Plot}\label{economic-loss-plot}

\begin{Shaded}
\begin{Highlighting}[]
\NormalTok{EL\_evolution }\OtherTok{=}\NormalTok{ DIIM\_model}\SpecialCharTok{$}\NormalTok{EL\_evolution}
\NormalTok{max\_econ\_loss }\OtherTok{\textless{}{-}} \FunctionTok{apply}\NormalTok{(EL\_evolution, }\DecValTok{1}\NormalTok{, max)}
\NormalTok{sorted\_indices }\OtherTok{\textless{}{-}} \FunctionTok{order}\NormalTok{(max\_econ\_loss, }\AttributeTok{decreasing =} \ConstantTok{TRUE}\NormalTok{)}
\NormalTok{sorted\_econ\_loss }\OtherTok{\textless{}{-}}\NormalTok{ EL\_evolution[sorted\_indices, ]}


\NormalTok{num\_sectors }\OtherTok{\textless{}{-}} \FunctionTok{nrow}\NormalTok{(sorted\_econ\_loss)}
\FunctionTok{matplot}\NormalTok{(}\FunctionTok{t}\NormalTok{(sorted\_econ\_loss), }\AttributeTok{type =} \StringTok{\textquotesingle{}l\textquotesingle{}}\NormalTok{, }\AttributeTok{lty =} \DecValTok{1}\NormalTok{, }\AttributeTok{col =} \FunctionTok{rainbow}\NormalTok{(num\_sectors),}
        \AttributeTok{xlab =} \StringTok{\textquotesingle{}Days\textquotesingle{}}\NormalTok{, }\AttributeTok{ylab =} \StringTok{\textquotesingle{}Economic Loss\textquotesingle{}}\NormalTok{,}
        \AttributeTok{main =} \StringTok{\textquotesingle{}Economic Loss Evolution\textquotesingle{}}\NormalTok{)}
\FunctionTok{legend}\NormalTok{(}\StringTok{\textquotesingle{}topright\textquotesingle{}}\NormalTok{, }\AttributeTok{legend =} \FunctionTok{paste}\NormalTok{(}\StringTok{\textquotesingle{}Sector\textquotesingle{}}\NormalTok{, sorted\_indices), }\AttributeTok{col =} \FunctionTok{rainbow}\NormalTok{(num\_sectors), }\AttributeTok{lty =} \DecValTok{1}\NormalTok{, }\AttributeTok{cex =} \FloatTok{0.6}\NormalTok{)}
\end{Highlighting}
\end{Shaded}

\includegraphics{fyp_files/figure-pdf/unnamed-chunk-4-1.pdf}

\subsection{Sensitivity Analysis}\label{sensitivity-analysis}

\subsubsection{Impact of altering lockdown
duration}\label{impact-of-altering-lockdown-duration}

\begin{Shaded}
\begin{Highlighting}[]
\NormalTok{output\_40\_days }\OtherTok{=} \FunctionTok{DIIM}\NormalTok{(q0,A\_star,c\_star,x,}\AttributeTok{lockdown\_duration=}\DecValTok{40}\NormalTok{, }\AttributeTok{total\_duration=}\DecValTok{751}\NormalTok{)}
\NormalTok{economics\_loss\_40\_days }\OtherTok{=} \FunctionTok{as.matrix}\NormalTok{(output\_40\_days}\SpecialCharTok{$}\NormalTok{EL\_end)}

\NormalTok{output\_55\_days }\OtherTok{=} \FunctionTok{DIIM}\NormalTok{(q0,A\_star,c\_star,x,}\AttributeTok{lockdown\_duration=}\DecValTok{55}\NormalTok{, }\AttributeTok{total\_duration=}\DecValTok{751}\NormalTok{)}
\NormalTok{economics\_loss\_55\_days }\OtherTok{=} \FunctionTok{as.matrix}\NormalTok{(output\_55\_days}\SpecialCharTok{$}\NormalTok{EL\_end)}

\NormalTok{output\_70\_days }\OtherTok{=} \FunctionTok{DIIM}\NormalTok{(q0,A\_star,c\_star,x,}\AttributeTok{lockdown\_duration=}\DecValTok{70}\NormalTok{, }\AttributeTok{total\_duration=}\DecValTok{751}\NormalTok{)}
\NormalTok{economics\_loss\_70\_days }\OtherTok{=} \FunctionTok{as.matrix}\NormalTok{(output\_70\_days}\SpecialCharTok{$}\NormalTok{EL\_end)}
\end{Highlighting}
\end{Shaded}

\subsection{Risk Management Scenario
Analysis}\label{risk-management-scenario-analysis}

\subsubsection{Scenario 1: assume adopt no risk management policy(set
policy option j =
0)}\label{scenario-1-assume-adopt-no-risk-management-policyset-policy-option-j-0}

In the first hypothetical scenario, assume that policy option j = 0,
that is, no risk management measures are taken, and all inoperability
and economic losses come from the calculation of the DIIM model result
in this paper, which is the real shock occurrence under the DIIM model.
Therefore, the economic loss after taking risk management is equal to
the economic loss without taking management measures and the execution
cost is 0. At the same time, we set scenario 1 as the baseline scenario
to facilitate subsequent analysis:

\begin{equation}\phantomsection\label{eq-1}{
\delta_j = \Gamma_{w[0]} - \Gamma_{w[j]} - \gamma_j = 0\,\text{ }
}\end{equation}

\subsubsection{Scenario 2: assume adopt risk management policy (set
policy option j =
1)}\label{scenario-2-assume-adopt-risk-management-policy-set-policy-option-j-1}

\begin{Shaded}
\begin{Highlighting}[]
\CommentTok{\# evolution of the inoperability under policy 1 where the inoperability is 95\% of the original inoperability}
\NormalTok{policy\_j\_0 }\OtherTok{=} \FunctionTok{DIIM}\NormalTok{(q0, A\_star,c\_star,x,}\AttributeTok{lockdown\_duration=}\DecValTok{55}\NormalTok{, }\AttributeTok{total\_duration=}\DecValTok{751}\NormalTok{,}\AttributeTok{risk\_management=}\DecValTok{1}\NormalTok{)}
\NormalTok{policy\_j\_1 }\OtherTok{=} \FunctionTok{DIIM}\NormalTok{(q0, A\_star,c\_star,x,}\AttributeTok{lockdown\_duration=}\DecValTok{55}\NormalTok{, }\AttributeTok{total\_duration=}\DecValTok{751}\NormalTok{,}\AttributeTok{risk\_management=}\FloatTok{0.95}\NormalTok{)}

\NormalTok{policy\_j\_0}\SpecialCharTok{$}\NormalTok{total\_economic\_loss}
\end{Highlighting}
\end{Shaded}

\begin{verbatim}
[1] 2026.506
\end{verbatim}

\begin{Shaded}
\begin{Highlighting}[]
\NormalTok{policy\_j\_1}\SpecialCharTok{$}\NormalTok{total\_economic\_loss}
\end{Highlighting}
\end{Shaded}

\begin{verbatim}
[1] 2103.948
\end{verbatim}

In this scenario, the paper assumed risk management policy 1 that the
government will spend 1 billion SGD in advance for risk control to
prevent the impact of the sudden spread of the pandemic on the economic
system as much as possible. Moreover, suppose that this advanced risk
management policy will reduce the impact of the pandemic by 5\% every
day, and when it acts on the inoperability level, it will only suffer
the impact of the 95\% degree of original inoperability level each day.
Bringing the changed degree of inoperability back into the DIIM model
for further analysis, it is found that under the implementation of risk
management policy 1, the overall economic loss is 725.3833 million SGD,
compared with the previous loss of 11181.38 million SGD, the overall
loss is reduced by 10456 million SGD. Therefore, the net benefit is 9456
million SGD, and the cost-benefit ratio is 0.0956389. The specific
calculation results are as follows:

\[
\begin{align}\delta_1 &= \Gamma_{w[0]} - \Gamma_{w[1]} - \gamma_1 = 26967.3 - 2323.01 - 1000 = 23644.29\, \textit{million SGD}  \\\lambda_{12} &= \frac{\gamma_1}{\Gamma_{w[0]} - \Gamma_{w[1]}} = \frac{1000}{26967.3 - 2323.01} = 0.04057735 \end{align}
\]

\subsubsection{Scenario 3: assume adopt risk management policy (set
policy option j =
2)}\label{scenario-3-assume-adopt-risk-management-policy-set-policy-option-j-2}

\begin{Shaded}
\begin{Highlighting}[]
\NormalTok{policy\_j\_3 }\OtherTok{=} \FunctionTok{DIIM}\NormalTok{(q0, A\_star,c\_star,x,}\AttributeTok{lockdown\_duration=}\DecValTok{55{-}20}\NormalTok{, }\AttributeTok{total\_duration=}\DecValTok{751}\NormalTok{,}\AttributeTok{risk\_management=}\DecValTok{1}\NormalTok{)}
\NormalTok{policy\_j\_3}\SpecialCharTok{$}\NormalTok{total\_economic\_loss}
\end{Highlighting}
\end{Shaded}

\begin{verbatim}
[1] 2026.506
\end{verbatim}

Based on risk management policy 2, it assumed that an additional 1
billion SGD investment cost will be added to the expenditure of early
prevention and control, resulting in a total investment cost of 2
billion SGD. This approach is to prevent the current and future
cascading effects of a wider pandemic on Singapore and to predict the
current year's budget in advance for risk prevention and control. Based
on this, the paper wants to calculate the impact of the additional 1
billion SGD of investment cost on the impact of the Singapore pandemic,
assuming that this additional expenditure cost can reducing the lockdown
period from the initial 55 days to 30 days. The overall economic loss
will be reduced from 11181.38 million SGD without risk management
measures to 7814.6 million SGD. Therefore, the net benefit is 1366.78
million SGD, the cost-benefit ratio is 0.5940394. The specific
calculation process is as follows:

\[
\begin{align*}\delta_2 &= \Gamma_{w[0]} - \Gamma_{w[2]} - \gamma_2 = 26967.3 - 17564.99 - 2000 = 7402.31\, \textit{million SGD} \\\lambda_{12} &= \frac{\gamma_2}{\Gamma_{w[0]} - \Gamma_{w[2]}} = \frac{2000}{26967.3 - 17564.99} = 0.2127137\end{align*}
\]

For the risk control management 1, the \(\lambda_{12}\) is 0.0956389
which indicates the government could obtain 1 SGD of benefit for each
0.0956389 SGD of investment cost. As with risk management policy 2, in
which the \(\lambda_{12}\) is 0.5940394, the government could obtain 1
SGD of benefit for each 0.5940394 SGD of investment cost. Compared with
these two policies, the first risk management measure is better, that
is, it has a smaller investment cost under the unit benefit.

\section{Comparison between ML and Traditional Key Sectors
Identified}\label{comparison-between-ml-and-traditional-key-sectors-identified}

We assume that the risk management policy is able to reduce the initial
inoperability of the key sectors by 0.9 of its original

\begin{Shaded}
\begin{Highlighting}[]
\CommentTok{\# ml\_key\_sectors = c(3,2,5,8,10,4,7)}
\CommentTok{\# trad\_key\_sectors = c(2,3,5,12,8,9,10)}

\NormalTok{lockdown\_duration\_vals }\OtherTok{=} \FunctionTok{c}\NormalTok{(}\DecValTok{10}\NormalTok{,}\DecValTok{20}\NormalTok{,}\DecValTok{30}\NormalTok{,}\DecValTok{40}\NormalTok{)}
\NormalTok{total\_duration\_vals }\OtherTok{=} \FunctionTok{c}\NormalTok{(}\DecValTok{300}\NormalTok{,}\DecValTok{400}\NormalTok{,}\DecValTok{500}\NormalTok{,}\DecValTok{600}\NormalTok{)}

\NormalTok{nsim }\OtherTok{=} \FunctionTok{length}\NormalTok{(lockdown\_duration\_vals) }\SpecialCharTok{*} \FunctionTok{length}\NormalTok{(total\_duration\_vals)}

\NormalTok{col\_names }\OtherTok{=} \FunctionTok{c}\NormalTok{(}\StringTok{"lockdown\_duration"}\NormalTok{,}
    \StringTok{"total\_duration"}\NormalTok{,}
    \StringTok{"model\_tot\_econ\_loss"}\NormalTok{,}
    \StringTok{"model\_diim\_tot\_econ\_loss"}\NormalTok{,}
    \StringTok{"model\_ml\_tot\_econ\_loss"}\NormalTok{)}

\NormalTok{sim\_matrix }\OtherTok{=} \FunctionTok{matrix}\NormalTok{(}\AttributeTok{data=}\ConstantTok{NA}\NormalTok{, }\AttributeTok{nrow=}\NormalTok{nsim , }\AttributeTok{ncol=}\DecValTok{5}\NormalTok{)}
\FunctionTok{colnames}\NormalTok{(sim\_matrix) }\OtherTok{=}\NormalTok{ col\_names}

\NormalTok{row\_idx }\OtherTok{=} \DecValTok{1}
\ControlFlowTok{for}\NormalTok{ (l\_duration }\ControlFlowTok{in}\NormalTok{ lockdown\_duration\_vals)\{}
  \ControlFlowTok{for}\NormalTok{ (t\_duration }\ControlFlowTok{in}\NormalTok{ total\_duration\_vals)\{}
\NormalTok{    res }\OtherTok{=} \FunctionTok{simulation\_ml\_vs\_diim}\NormalTok{(q0, A\_star,c\_star,x,}\AttributeTok{lockdown\_duration=}\NormalTok{l\_duration, }\AttributeTok{total\_duration=}\NormalTok{t\_duration)}
\NormalTok{    res }\OtherTok{=} \FunctionTok{matrix}\NormalTok{(}\FunctionTok{unlist}\NormalTok{(res), }\AttributeTok{ncol=}\DecValTok{5}\NormalTok{)}
\NormalTok{    sim\_matrix[row\_idx,] }\OtherTok{=}\NormalTok{ res}
    
\NormalTok{    row\_idx }\OtherTok{=}\NormalTok{ row\_idx }\SpecialCharTok{+} \DecValTok{1}
\NormalTok{  \}}
\NormalTok{\}}
\end{Highlighting}
\end{Shaded}

\section{Comparing No Intervention, DIIM, and ML Economic Loss
Models}\label{comparing-no-intervention-diim-and-ml-economic-loss-models}

\begin{Shaded}
\begin{Highlighting}[]
\NormalTok{df }\OtherTok{\textless{}{-}} \FunctionTok{as.data.frame}\NormalTok{(sim\_matrix)}
\NormalTok{df}\SpecialCharTok{$}\NormalTok{lockdown\_duration }\OtherTok{\textless{}{-}} \FunctionTok{factor}\NormalTok{(df}\SpecialCharTok{$}\NormalTok{lockdown\_duration)}
\end{Highlighting}
\end{Shaded}

\begin{Shaded}
\begin{Highlighting}[]
\NormalTok{df\_long }\OtherTok{\textless{}{-}} \FunctionTok{melt}\NormalTok{(df, }
                \AttributeTok{id.vars =} \FunctionTok{c}\NormalTok{(}\StringTok{"lockdown\_duration"}\NormalTok{, }\StringTok{"total\_duration"}\NormalTok{),}
                \AttributeTok{variable.name =} \StringTok{"model"}\NormalTok{,}
                \AttributeTok{value.name =} \StringTok{"economic\_loss"}\NormalTok{)}

\NormalTok{plot1 }\OtherTok{\textless{}{-}} \FunctionTok{ggplot}\NormalTok{(df\_long, }\FunctionTok{aes}\NormalTok{(}\AttributeTok{x =}\NormalTok{ total\_duration, }\AttributeTok{y =}\NormalTok{ economic\_loss, }
                              \AttributeTok{color =}\NormalTok{ model, }\AttributeTok{linetype =}\NormalTok{ model)) }\SpecialCharTok{+}
  \FunctionTok{geom\_line}\NormalTok{(}\AttributeTok{size =} \DecValTok{1}\NormalTok{) }\SpecialCharTok{+}
  \FunctionTok{geom\_point}\NormalTok{(}\AttributeTok{size =} \DecValTok{2}\NormalTok{) }\SpecialCharTok{+}
  \FunctionTok{facet\_wrap}\NormalTok{(}\SpecialCharTok{\textasciitilde{}}\NormalTok{lockdown\_duration, }\AttributeTok{labeller =} \FunctionTok{labeller}\NormalTok{(}
    \AttributeTok{lockdown\_duration =} \FunctionTok{c}\NormalTok{(}\StringTok{"10"} \OtherTok{=} \StringTok{"Lockdown: 10 days"}\NormalTok{,}
                          \StringTok{"20"} \OtherTok{=} \StringTok{"Lockdown: 20 days"}\NormalTok{,}
                          \StringTok{"30"} \OtherTok{=} \StringTok{"Lockdown: 30 days"}\NormalTok{,}
                          \StringTok{"40"} \OtherTok{=} \StringTok{"Lockdown: 40 days"}\NormalTok{))) }\SpecialCharTok{+}
  \FunctionTok{scale\_color\_manual}\NormalTok{(}\AttributeTok{name =} \StringTok{"Model"}\NormalTok{,}
                     \AttributeTok{values =} \FunctionTok{c}\NormalTok{(}\StringTok{"model\_tot\_econ\_loss"} \OtherTok{=} \StringTok{"\#E74C3C"}\NormalTok{,}
                               \StringTok{"model\_diim\_tot\_econ\_loss"} \OtherTok{=} \StringTok{"\#3498DB"}\NormalTok{,}
                               \StringTok{"model\_ml\_tot\_econ\_loss"} \OtherTok{=} \StringTok{"\#2ECC71"}\NormalTok{),}
                     \AttributeTok{labels =} \FunctionTok{c}\NormalTok{(}\StringTok{"model\_tot\_econ\_loss"} \OtherTok{=} \StringTok{"No Intervention"}\NormalTok{,}
                               \StringTok{"model\_diim\_tot\_econ\_loss"} \OtherTok{=} \StringTok{"DIIM"}\NormalTok{,}
                               \StringTok{"model\_ml\_tot\_econ\_loss"} \OtherTok{=} \StringTok{"ML Model"}\NormalTok{)) }\SpecialCharTok{+}
  \FunctionTok{scale\_linetype\_manual}\NormalTok{(}\AttributeTok{name =} \StringTok{"Model"}\NormalTok{,}
                       \AttributeTok{values =} \FunctionTok{c}\NormalTok{(}\StringTok{"model\_tot\_econ\_loss"} \OtherTok{=} \StringTok{"solid"}\NormalTok{,}
                                 \StringTok{"model\_diim\_tot\_econ\_loss"} \OtherTok{=} \StringTok{"dashed"}\NormalTok{,}
                                 \StringTok{"model\_ml\_tot\_econ\_loss"} \OtherTok{=} \StringTok{"dotted"}\NormalTok{),}
                       \AttributeTok{labels =} \FunctionTok{c}\NormalTok{(}\StringTok{"model\_tot\_econ\_loss"} \OtherTok{=} \StringTok{"No Intervention"}\NormalTok{,}
                                 \StringTok{"model\_diim\_tot\_econ\_loss"} \OtherTok{=} \StringTok{"DIIM"}\NormalTok{,}
                                 \StringTok{"model\_ml\_tot\_econ\_loss"} \OtherTok{=} \StringTok{"ML Model"}\NormalTok{)) }\SpecialCharTok{+}
  \FunctionTok{labs}\NormalTok{(}\AttributeTok{title =} \StringTok{"Economic Loss Comparison: Impact of Total Duration on Models"}\NormalTok{,}
       \AttributeTok{x =} \StringTok{"Total Duration (days)"}\NormalTok{,}
       \AttributeTok{y =} \StringTok{"Economic Loss ($)"}\NormalTok{) }\SpecialCharTok{+}
  \FunctionTok{theme\_minimal}\NormalTok{() }\SpecialCharTok{+}
  \FunctionTok{theme}\NormalTok{(}\AttributeTok{plot.title =} \FunctionTok{element\_text}\NormalTok{(}\AttributeTok{face =} \StringTok{"bold"}\NormalTok{, }\AttributeTok{size =} \DecValTok{14}\NormalTok{),}
        \AttributeTok{legend.position =} \StringTok{"bottom"}\NormalTok{,}
        \AttributeTok{panel.grid.major =} \FunctionTok{element\_line}\NormalTok{(}\AttributeTok{color =} \StringTok{"gray90"}\NormalTok{))}
\end{Highlighting}
\end{Shaded}

\begin{verbatim}
Warning: Using `size` aesthetic for lines was deprecated in ggplot2 3.4.0.
i Please use `linewidth` instead.
\end{verbatim}

\begin{Shaded}
\begin{Highlighting}[]
\FunctionTok{print}\NormalTok{(plot1)}
\end{Highlighting}
\end{Shaded}

\includegraphics{fyp_files/figure-pdf/unnamed-chunk-10-1.pdf}

\begin{Shaded}
\begin{Highlighting}[]
\CommentTok{\# No Intervention Model Heatmap}
\NormalTok{heatmap\_data1 }\OtherTok{\textless{}{-}}\NormalTok{ df }\SpecialCharTok{\%\textgreater{}\%}
  \FunctionTok{select}\NormalTok{(lockdown\_duration, total\_duration, model\_tot\_econ\_loss) }\SpecialCharTok{\%\textgreater{}\%}
  \FunctionTok{pivot\_wider}\NormalTok{(}\AttributeTok{names\_from =}\NormalTok{ total\_duration, }\AttributeTok{values\_from =}\NormalTok{ model\_tot\_econ\_loss)}

\NormalTok{heatmap\_matrix1 }\OtherTok{\textless{}{-}} \FunctionTok{as.matrix}\NormalTok{(heatmap\_data1[, }\SpecialCharTok{{-}}\DecValTok{1}\NormalTok{])}
\FunctionTok{rownames}\NormalTok{(heatmap\_matrix1) }\OtherTok{\textless{}{-}}\NormalTok{ heatmap\_data1}\SpecialCharTok{$}\NormalTok{lockdown\_duration}

\CommentTok{\# DIIM Model Heatmap}
\NormalTok{heatmap\_data2 }\OtherTok{\textless{}{-}}\NormalTok{ df }\SpecialCharTok{\%\textgreater{}\%}
  \FunctionTok{select}\NormalTok{(lockdown\_duration, total\_duration, model\_diim\_tot\_econ\_loss) }\SpecialCharTok{\%\textgreater{}\%}
  \FunctionTok{pivot\_wider}\NormalTok{(}\AttributeTok{names\_from =}\NormalTok{ total\_duration, }\AttributeTok{values\_from =}\NormalTok{ model\_diim\_tot\_econ\_loss)}

\NormalTok{heatmap\_matrix2 }\OtherTok{\textless{}{-}} \FunctionTok{as.matrix}\NormalTok{(heatmap\_data2[, }\SpecialCharTok{{-}}\DecValTok{1}\NormalTok{])}
\FunctionTok{rownames}\NormalTok{(heatmap\_matrix2) }\OtherTok{\textless{}{-}}\NormalTok{ heatmap\_data2}\SpecialCharTok{$}\NormalTok{lockdown\_duration}

\CommentTok{\# ML Model Heatmap}
\NormalTok{heatmap\_data3 }\OtherTok{\textless{}{-}}\NormalTok{ df }\SpecialCharTok{\%\textgreater{}\%}
  \FunctionTok{select}\NormalTok{(lockdown\_duration, total\_duration, model\_ml\_tot\_econ\_loss) }\SpecialCharTok{\%\textgreater{}\%}
  \FunctionTok{pivot\_wider}\NormalTok{(}\AttributeTok{names\_from =}\NormalTok{ total\_duration, }\AttributeTok{values\_from =}\NormalTok{ model\_ml\_tot\_econ\_loss)}

\NormalTok{heatmap\_matrix3 }\OtherTok{\textless{}{-}} \FunctionTok{as.matrix}\NormalTok{(heatmap\_data3[, }\SpecialCharTok{{-}}\DecValTok{1}\NormalTok{])}
\FunctionTok{rownames}\NormalTok{(heatmap\_matrix3) }\OtherTok{\textless{}{-}}\NormalTok{ heatmap\_data3}\SpecialCharTok{$}\NormalTok{lockdown\_duration}

\CommentTok{\# Create heatmap data frame for ggplot2}
\NormalTok{heatmap\_long }\OtherTok{\textless{}{-}}\NormalTok{ df\_long }\SpecialCharTok{\%\textgreater{}\%}
  \FunctionTok{mutate}\NormalTok{(}\AttributeTok{lockdown\_duration =} \FunctionTok{as.character}\NormalTok{(lockdown\_duration),}
         \AttributeTok{total\_duration =} \FunctionTok{as.character}\NormalTok{(total\_duration))}

\NormalTok{plot2 }\OtherTok{\textless{}{-}} \FunctionTok{ggplot}\NormalTok{(heatmap\_long, }\FunctionTok{aes}\NormalTok{(}\AttributeTok{x =}\NormalTok{ total\_duration, }\AttributeTok{y =}\NormalTok{ lockdown\_duration, }
                                   \AttributeTok{fill =}\NormalTok{ economic\_loss)) }\SpecialCharTok{+}
  \FunctionTok{geom\_tile}\NormalTok{(}\AttributeTok{color =} \StringTok{"white"}\NormalTok{, }\AttributeTok{size =} \FloatTok{0.5}\NormalTok{) }\SpecialCharTok{+}
  \FunctionTok{facet\_wrap}\NormalTok{(}\SpecialCharTok{\textasciitilde{}}\NormalTok{model, }\AttributeTok{labeller =} \FunctionTok{labeller}\NormalTok{(}
    \AttributeTok{model =} \FunctionTok{c}\NormalTok{(}\StringTok{"model\_tot\_econ\_loss"} \OtherTok{=} \StringTok{"No Intervention"}\NormalTok{,}
             \StringTok{"model\_diim\_tot\_econ\_loss"} \OtherTok{=} \StringTok{"DIIM Model"}\NormalTok{,}
             \StringTok{"model\_ml\_tot\_econ\_loss"} \OtherTok{=} \StringTok{"ML Model"}\NormalTok{))) }\SpecialCharTok{+}
  \FunctionTok{scale\_fill\_gradient}\NormalTok{(}\AttributeTok{low =} \StringTok{"\#ECF0F1"}\NormalTok{, }\AttributeTok{high =} \StringTok{"\#C0392B"}\NormalTok{,}
                      \AttributeTok{name =} \StringTok{"Economic}\SpecialCharTok{\textbackslash{}n}\StringTok{Loss ($)"}\NormalTok{) }\SpecialCharTok{+}
  \FunctionTok{labs}\NormalTok{(}\AttributeTok{title =} \StringTok{"Economic Loss Heatmap Across Models"}\NormalTok{,}
       \AttributeTok{x =} \StringTok{"Total Duration (days)"}\NormalTok{,}
       \AttributeTok{y =} \StringTok{"Lockdown Duration (days)"}\NormalTok{) }\SpecialCharTok{+}
  \FunctionTok{theme\_minimal}\NormalTok{() }\SpecialCharTok{+}
  \FunctionTok{theme}\NormalTok{(}\AttributeTok{plot.title =} \FunctionTok{element\_text}\NormalTok{(}\AttributeTok{face =} \StringTok{"bold"}\NormalTok{, }\AttributeTok{size =} \DecValTok{14}\NormalTok{),}
        \AttributeTok{axis.text =} \FunctionTok{element\_text}\NormalTok{(}\AttributeTok{size =} \DecValTok{10}\NormalTok{),}
        \AttributeTok{strip.text =} \FunctionTok{element\_text}\NormalTok{(}\AttributeTok{face =} \StringTok{"bold"}\NormalTok{))}

\FunctionTok{print}\NormalTok{(plot2)}
\end{Highlighting}
\end{Shaded}

\includegraphics{fyp_files/figure-pdf/unnamed-chunk-11-1.pdf}

\begin{Shaded}
\begin{Highlighting}[]
\CommentTok{\# Calculate improvements}

\NormalTok{df}\SpecialCharTok{$}\NormalTok{ml\_improvement }\OtherTok{\textless{}{-}}\NormalTok{ df}\SpecialCharTok{$}\NormalTok{model\_tot\_econ\_loss }\SpecialCharTok{{-}}\NormalTok{ df}\SpecialCharTok{$}\NormalTok{model\_ml\_tot\_econ\_loss}
\NormalTok{df}\SpecialCharTok{$}\NormalTok{diim\_improvement }\OtherTok{\textless{}{-}}\NormalTok{ df}\SpecialCharTok{$}\NormalTok{model\_tot\_econ\_loss }\SpecialCharTok{{-}}\NormalTok{ df}\SpecialCharTok{$}\NormalTok{model\_diim\_tot\_econ\_loss}
\NormalTok{df}\SpecialCharTok{$}\NormalTok{ml\_pct\_improvement }\OtherTok{\textless{}{-}}\NormalTok{ (df}\SpecialCharTok{$}\NormalTok{ml\_improvement }\SpecialCharTok{/}\NormalTok{ df}\SpecialCharTok{$}\NormalTok{model\_tot\_econ\_loss) }\SpecialCharTok{*} \DecValTok{100}
\NormalTok{df}\SpecialCharTok{$}\NormalTok{diim\_pct\_improvement }\OtherTok{\textless{}{-}}\NormalTok{ (df}\SpecialCharTok{$}\NormalTok{diim\_improvement }\SpecialCharTok{/}\NormalTok{ df}\SpecialCharTok{$}\NormalTok{model\_tot\_econ\_loss) }\SpecialCharTok{*} \DecValTok{100}

\NormalTok{df\_improvement }\OtherTok{\textless{}{-}}\NormalTok{ df }\SpecialCharTok{\%\textgreater{}\%}
  \FunctionTok{select}\NormalTok{(lockdown\_duration, total\_duration, ml\_improvement, diim\_improvement) }\SpecialCharTok{\%\textgreater{}\%}
  \FunctionTok{melt}\NormalTok{(}\AttributeTok{id.vars =} \FunctionTok{c}\NormalTok{(}\StringTok{"lockdown\_duration"}\NormalTok{, }\StringTok{"total\_duration"}\NormalTok{),}
       \AttributeTok{variable.name =} \StringTok{"model"}\NormalTok{,}
       \AttributeTok{value.name =} \StringTok{"loss\_reduction"}\NormalTok{)}

\NormalTok{plot4 }\OtherTok{\textless{}{-}} \FunctionTok{ggplot}\NormalTok{(df\_improvement, }\FunctionTok{aes}\NormalTok{(}\AttributeTok{x =} \FunctionTok{factor}\NormalTok{(total\_duration), }\AttributeTok{y =}\NormalTok{ loss\_reduction, }
                                     \AttributeTok{fill =}\NormalTok{ model)) }\SpecialCharTok{+}
  \FunctionTok{geom\_bar}\NormalTok{(}\AttributeTok{stat =} \StringTok{"identity"}\NormalTok{, }\AttributeTok{position =} \StringTok{"dodge"}\NormalTok{) }\SpecialCharTok{+}
  \FunctionTok{facet\_wrap}\NormalTok{(}\SpecialCharTok{\textasciitilde{}}\NormalTok{lockdown\_duration, }\AttributeTok{labeller =} \FunctionTok{labeller}\NormalTok{(}
    \AttributeTok{lockdown\_duration =} \FunctionTok{c}\NormalTok{(}\StringTok{"10"} \OtherTok{=} \StringTok{"Lockdown: 10 days"}\NormalTok{,}
                          \StringTok{"20"} \OtherTok{=} \StringTok{"Lockdown: 20 days"}\NormalTok{,}
                          \StringTok{"30"} \OtherTok{=} \StringTok{"Lockdown: 30 days"}\NormalTok{,}
                          \StringTok{"40"} \OtherTok{=} \StringTok{"Lockdown: 40 days"}\NormalTok{))) }\SpecialCharTok{+}
  \FunctionTok{scale\_fill\_manual}\NormalTok{(}\AttributeTok{name =} \StringTok{"Model"}\NormalTok{,}
                   \AttributeTok{values =} \FunctionTok{c}\NormalTok{(}\StringTok{"ml\_improvement"} \OtherTok{=} \StringTok{"\#2ECC71"}\NormalTok{,}
                             \StringTok{"diim\_improvement"} \OtherTok{=} \StringTok{"\#3498DB"}\NormalTok{),}
                   \AttributeTok{labels =} \FunctionTok{c}\NormalTok{(}\StringTok{"ml\_improvement"} \OtherTok{=} \StringTok{"ML Model"}\NormalTok{,}
                             \StringTok{"diim\_improvement"} \OtherTok{=} \StringTok{"DIIM"}\NormalTok{)) }\SpecialCharTok{+}
  \FunctionTok{labs}\NormalTok{(}\AttributeTok{title =} \StringTok{"Economic Loss Reduction: Model Improvements vs No Intervention"}\NormalTok{,}
       \AttributeTok{x =} \StringTok{"Total Duration (days)"}\NormalTok{,}
       \AttributeTok{y =} \StringTok{"Loss Reduction ($)"}\NormalTok{) }\SpecialCharTok{+}
  \FunctionTok{theme\_minimal}\NormalTok{() }\SpecialCharTok{+}
  \FunctionTok{theme}\NormalTok{(}\AttributeTok{plot.title =} \FunctionTok{element\_text}\NormalTok{(}\AttributeTok{face =} \StringTok{"bold"}\NormalTok{, }\AttributeTok{size =} \DecValTok{14}\NormalTok{),}
        \AttributeTok{panel.grid.major.y =} \FunctionTok{element\_line}\NormalTok{(}\AttributeTok{color =} \StringTok{"gray90"}\NormalTok{))}

\FunctionTok{print}\NormalTok{(plot4)}
\end{Highlighting}
\end{Shaded}

\includegraphics{fyp_files/figure-pdf/unnamed-chunk-12-1.pdf}

\begin{Shaded}
\begin{Highlighting}[]
\NormalTok{df}\SpecialCharTok{$}\NormalTok{ml\_pct\_vs\_no\_int }\OtherTok{\textless{}{-}}\NormalTok{ (df}\SpecialCharTok{$}\NormalTok{ml\_improvement }\SpecialCharTok{/}\NormalTok{ df}\SpecialCharTok{$}\NormalTok{model\_tot\_econ\_loss) }\SpecialCharTok{*} \DecValTok{100}

\NormalTok{pct\_improvement\_data }\OtherTok{\textless{}{-}}\NormalTok{ df }\SpecialCharTok{\%\textgreater{}\%}
  \FunctionTok{select}\NormalTok{(lockdown\_duration, total\_duration, ml\_pct\_vs\_no\_int) }\SpecialCharTok{\%\textgreater{}\%}
  \FunctionTok{mutate}\NormalTok{(}\AttributeTok{lockdown\_duration =} \FunctionTok{as.character}\NormalTok{(lockdown\_duration),}
         \AttributeTok{total\_duration =} \FunctionTok{as.character}\NormalTok{(total\_duration))}

\NormalTok{plot5 }\OtherTok{\textless{}{-}} \FunctionTok{ggplot}\NormalTok{(pct\_improvement\_data, }\FunctionTok{aes}\NormalTok{(}\AttributeTok{x =}\NormalTok{ total\_duration, }\AttributeTok{y =}\NormalTok{ lockdown\_duration, }
                                           \AttributeTok{fill =}\NormalTok{ ml\_pct\_vs\_no\_int)) }\SpecialCharTok{+}
  \FunctionTok{geom\_tile}\NormalTok{(}\AttributeTok{color =} \StringTok{"white"}\NormalTok{, }\AttributeTok{size =} \FloatTok{0.8}\NormalTok{) }\SpecialCharTok{+}
  \FunctionTok{geom\_text}\NormalTok{(}\FunctionTok{aes}\NormalTok{(}\AttributeTok{label =} \FunctionTok{sprintf}\NormalTok{(}\StringTok{"\%.2f\%\%"}\NormalTok{, ml\_pct\_vs\_no\_int)), }
            \AttributeTok{color =} \StringTok{"black"}\NormalTok{, }\AttributeTok{size =} \DecValTok{3}\NormalTok{) }\SpecialCharTok{+}
  \FunctionTok{scale\_fill\_gradient}\NormalTok{(}\AttributeTok{low =} \StringTok{"\#E8F8F5"}\NormalTok{, }\AttributeTok{high =} \StringTok{"\#27AE60"}\NormalTok{,}
                      \AttributeTok{name =} \StringTok{"\% Improvement"}\NormalTok{) }\SpecialCharTok{+}
  \FunctionTok{labs}\NormalTok{(}\AttributeTok{title =} \StringTok{"ML Model: Percentage Improvement vs No Intervention"}\NormalTok{,}
       \AttributeTok{x =} \StringTok{"Total Duration (days)"}\NormalTok{,}
       \AttributeTok{y =} \StringTok{"Lockdown Duration (days)"}\NormalTok{) }\SpecialCharTok{+}
  \FunctionTok{theme\_minimal}\NormalTok{() }\SpecialCharTok{+}
  \FunctionTok{theme}\NormalTok{(}\AttributeTok{plot.title =} \FunctionTok{element\_text}\NormalTok{(}\AttributeTok{face =} \StringTok{"bold"}\NormalTok{, }\AttributeTok{size =} \DecValTok{14}\NormalTok{),}
        \AttributeTok{axis.text =} \FunctionTok{element\_text}\NormalTok{(}\AttributeTok{size =} \DecValTok{11}\NormalTok{))}

\FunctionTok{print}\NormalTok{(plot5)}
\end{Highlighting}
\end{Shaded}

\includegraphics{fyp_files/figure-pdf/unnamed-chunk-13-1.pdf}

\begin{Shaded}
\begin{Highlighting}[]
\NormalTok{df}\SpecialCharTok{$}\NormalTok{ml\_pct\_vs\_no\_int }\OtherTok{\textless{}{-}}\NormalTok{ df}\SpecialCharTok{$}\NormalTok{ml\_pct\_improvement }\SpecialCharTok{{-}}\NormalTok{ df}\SpecialCharTok{$}\NormalTok{diim\_pct\_improvement}

\NormalTok{pct\_improvement\_data }\OtherTok{\textless{}{-}}\NormalTok{ df }\SpecialCharTok{\%\textgreater{}\%}
  \FunctionTok{select}\NormalTok{(lockdown\_duration, total\_duration, ml\_pct\_vs\_no\_int) }\SpecialCharTok{\%\textgreater{}\%}
  \FunctionTok{mutate}\NormalTok{(}\AttributeTok{lockdown\_duration =} \FunctionTok{as.character}\NormalTok{(lockdown\_duration),}
         \AttributeTok{total\_duration =} \FunctionTok{as.character}\NormalTok{(total\_duration))}

\NormalTok{plot5 }\OtherTok{\textless{}{-}} \FunctionTok{ggplot}\NormalTok{(pct\_improvement\_data, }\FunctionTok{aes}\NormalTok{(}\AttributeTok{x =}\NormalTok{ total\_duration, }\AttributeTok{y =}\NormalTok{ lockdown\_duration, }
                                           \AttributeTok{fill =}\NormalTok{ ml\_pct\_vs\_no\_int)) }\SpecialCharTok{+}
  \FunctionTok{geom\_tile}\NormalTok{(}\AttributeTok{color =} \StringTok{"white"}\NormalTok{, }\AttributeTok{size =} \FloatTok{0.8}\NormalTok{) }\SpecialCharTok{+}
  \FunctionTok{geom\_text}\NormalTok{(}\FunctionTok{aes}\NormalTok{(}\AttributeTok{label =} \FunctionTok{sprintf}\NormalTok{(}\StringTok{"\%.2f\%\%"}\NormalTok{, ml\_pct\_vs\_no\_int)), }
            \AttributeTok{color =} \StringTok{"black"}\NormalTok{, }\AttributeTok{size =} \DecValTok{3}\NormalTok{) }\SpecialCharTok{+}
  \FunctionTok{scale\_fill\_gradient}\NormalTok{(}\AttributeTok{low =} \StringTok{"\#E8F8F5"}\NormalTok{, }\AttributeTok{high =} \StringTok{"\#27AE60"}\NormalTok{,}
                      \AttributeTok{name =} \StringTok{"\% Improvement"}\NormalTok{) }\SpecialCharTok{+}
  \FunctionTok{labs}\NormalTok{(}\AttributeTok{title =} \StringTok{"Difference in Improvement Between DIIM and ML Model"}\NormalTok{,}
       \AttributeTok{x =} \StringTok{"Total Duration (days)"}\NormalTok{,}
       \AttributeTok{y =} \StringTok{"Lockdown Duration (days)"}\NormalTok{) }\SpecialCharTok{+}
  \FunctionTok{theme\_minimal}\NormalTok{() }\SpecialCharTok{+}
  \FunctionTok{theme}\NormalTok{(}\AttributeTok{plot.title =} \FunctionTok{element\_text}\NormalTok{(}\AttributeTok{face =} \StringTok{"bold"}\NormalTok{, }\AttributeTok{size =} \DecValTok{14}\NormalTok{),}
        \AttributeTok{axis.text =} \FunctionTok{element\_text}\NormalTok{(}\AttributeTok{size =} \DecValTok{11}\NormalTok{))}

\FunctionTok{print}\NormalTok{(plot5)}
\end{Highlighting}
\end{Shaded}

\includegraphics{fyp_files/figure-pdf/unnamed-chunk-14-1.pdf}

\begin{Shaded}
\begin{Highlighting}[]
\NormalTok{plot6 }\OtherTok{\textless{}{-}} \FunctionTok{ggplot}\NormalTok{(df, }\FunctionTok{aes}\NormalTok{(}\AttributeTok{x =} \FunctionTok{factor}\NormalTok{(total\_duration), }
                        \AttributeTok{y =}\NormalTok{ model\_ml\_tot\_econ\_loss,}
                        \AttributeTok{color =}\NormalTok{ lockdown\_duration,}
                        \AttributeTok{group =}\NormalTok{ lockdown\_duration)) }\SpecialCharTok{+}
  \FunctionTok{geom\_line}\NormalTok{(}\AttributeTok{size =} \DecValTok{1}\NormalTok{) }\SpecialCharTok{+}
  \FunctionTok{geom\_point}\NormalTok{(}\AttributeTok{size =} \DecValTok{3}\NormalTok{) }\SpecialCharTok{+}
  \FunctionTok{labs}\NormalTok{(}\AttributeTok{title =} \StringTok{"ML Model Economic Loss: Interaction Plot"}\NormalTok{,}
       \AttributeTok{subtitle =} \StringTok{"Shows combined effect of lockdown duration and total duration"}\NormalTok{,}
       \AttributeTok{x =} \StringTok{"Total Duration (days)"}\NormalTok{,}
       \AttributeTok{y =} \StringTok{"Economic Loss ($)"}\NormalTok{,}
       \AttributeTok{color =} \StringTok{"Lockdown}\SpecialCharTok{\textbackslash{}n}\StringTok{Duration (days)"}\NormalTok{) }\SpecialCharTok{+}
  \FunctionTok{theme\_minimal}\NormalTok{() }\SpecialCharTok{+}
  \FunctionTok{theme}\NormalTok{(}\AttributeTok{plot.title =} \FunctionTok{element\_text}\NormalTok{(}\AttributeTok{face =} \StringTok{"bold"}\NormalTok{, }\AttributeTok{size =} \DecValTok{14}\NormalTok{),}
        \AttributeTok{panel.grid.major =} \FunctionTok{element\_line}\NormalTok{(}\AttributeTok{color =} \StringTok{"gray90"}\NormalTok{))}

\FunctionTok{print}\NormalTok{(plot6)}
\end{Highlighting}
\end{Shaded}

\includegraphics{fyp_files/figure-pdf/unnamed-chunk-15-1.pdf}

\begin{Shaded}
\begin{Highlighting}[]
\FunctionTok{ggsave}\NormalTok{(}\StringTok{"plot6\_interaction\_plot.png"}\NormalTok{, plot6, }\AttributeTok{width =} \DecValTok{10}\NormalTok{, }\AttributeTok{height =} \DecValTok{6}\NormalTok{)}
\end{Highlighting}
\end{Shaded}

\begin{Shaded}
\begin{Highlighting}[]
\NormalTok{df\_long\_box }\OtherTok{\textless{}{-}}\NormalTok{ df\_long }\SpecialCharTok{\%\textgreater{}\%}
  \FunctionTok{mutate}\NormalTok{(}\AttributeTok{lockdown\_duration =} \FunctionTok{factor}\NormalTok{(lockdown\_duration))}

\NormalTok{plot7 }\OtherTok{\textless{}{-}} \FunctionTok{ggplot}\NormalTok{(df\_long\_box, }\FunctionTok{aes}\NormalTok{(}\AttributeTok{x =}\NormalTok{ model, }\AttributeTok{y =}\NormalTok{ economic\_loss, }\AttributeTok{fill =}\NormalTok{ model)) }\SpecialCharTok{+}
  \FunctionTok{geom\_boxplot}\NormalTok{(}\AttributeTok{alpha =} \FloatTok{0.7}\NormalTok{) }\SpecialCharTok{+}
  \FunctionTok{facet\_wrap}\NormalTok{(}\SpecialCharTok{\textasciitilde{}}\NormalTok{lockdown\_duration, }\AttributeTok{labeller =} \FunctionTok{labeller}\NormalTok{(}
    \AttributeTok{lockdown\_duration =} \FunctionTok{c}\NormalTok{(}\StringTok{"10"} \OtherTok{=} \StringTok{"Lockdown: 10 days"}\NormalTok{,}
                          \StringTok{"20"} \OtherTok{=} \StringTok{"Lockdown: 20 days"}\NormalTok{,}
                          \StringTok{"30"} \OtherTok{=} \StringTok{"Lockdown: 30 days"}\NormalTok{,}
                          \StringTok{"40"} \OtherTok{=} \StringTok{"Lockdown: 40 days"}\NormalTok{))) }\SpecialCharTok{+}
  \FunctionTok{scale\_fill\_manual}\NormalTok{(}\AttributeTok{values =} \FunctionTok{c}\NormalTok{(}\StringTok{"model\_tot\_econ\_loss"} \OtherTok{=} \StringTok{"\#E74C3C"}\NormalTok{,}
                               \StringTok{"model\_trad\_tot\_econ\_loss"} \OtherTok{=} \StringTok{"\#3498DB"}\NormalTok{,}
                               \StringTok{"model\_ml\_tot\_econ\_loss"} \OtherTok{=} \StringTok{"\#2ECC71"}\NormalTok{),}
                    \AttributeTok{labels =} \FunctionTok{c}\NormalTok{(}\StringTok{"model\_tot\_econ\_loss"} \OtherTok{=} \StringTok{"No Intervention"}\NormalTok{,}
                              \StringTok{"model\_trad\_tot\_econ\_loss"} \OtherTok{=} \StringTok{"Traditional"}\NormalTok{,}
                              \StringTok{"model\_ml\_tot\_econ\_loss"} \OtherTok{=} \StringTok{"ML Model"}\NormalTok{)) }\SpecialCharTok{+}
  \FunctionTok{scale\_x\_discrete}\NormalTok{(}\AttributeTok{labels =} \FunctionTok{c}\NormalTok{(}\StringTok{"model\_tot\_econ\_loss"} \OtherTok{=} \StringTok{"No}\SpecialCharTok{\textbackslash{}n}\StringTok{Intervention"}\NormalTok{,}
                             \StringTok{"model\_trad\_tot\_econ\_loss"} \OtherTok{=} \StringTok{"Traditional"}\NormalTok{,}
                             \StringTok{"model\_ml\_tot\_econ\_loss"} \OtherTok{=} \StringTok{"ML Model"}\NormalTok{)) }\SpecialCharTok{+}
  \FunctionTok{labs}\NormalTok{(}\AttributeTok{title =} \StringTok{"Distribution of Economic Loss Across Models"}\NormalTok{,}
       \AttributeTok{x =} \StringTok{"Model Type"}\NormalTok{,}
       \AttributeTok{y =} \StringTok{"Economic Loss ($)"}\NormalTok{) }\SpecialCharTok{+}
  \FunctionTok{guides}\NormalTok{(}\AttributeTok{fill =} \ConstantTok{FALSE}\NormalTok{) }\SpecialCharTok{+}
  \FunctionTok{theme\_minimal}\NormalTok{() }\SpecialCharTok{+}
  \FunctionTok{theme}\NormalTok{(}\AttributeTok{plot.title =} \FunctionTok{element\_text}\NormalTok{(}\AttributeTok{face =} \StringTok{"bold"}\NormalTok{, }\AttributeTok{size =} \DecValTok{14}\NormalTok{),}
        \AttributeTok{axis.text.x =} \FunctionTok{element\_text}\NormalTok{(}\AttributeTok{size =} \DecValTok{9}\NormalTok{))}
\end{Highlighting}
\end{Shaded}

\begin{verbatim}
Warning: The `<scale>` argument of `guides()` cannot be `FALSE`. Use "none" instead as
of ggplot2 3.3.4.
\end{verbatim}

\begin{Shaded}
\begin{Highlighting}[]
\FunctionTok{print}\NormalTok{(plot7)}
\end{Highlighting}
\end{Shaded}

\includegraphics{fyp_files/figure-pdf/unnamed-chunk-16-1.pdf}

\begin{Shaded}
\begin{Highlighting}[]
\CommentTok{\# summary\_stats \textless{}{-} df \%\textgreater{}\%}
\CommentTok{\#   select(lockdown\_duration, model\_tot\_econ\_loss, model\_trad\_tot\_econ\_loss, }
\CommentTok{\#          model\_ml\_tot\_econ\_loss, ml\_improvement, trad\_improvement, ml\_pct\_improvement) \%\textgreater{}\%}
\CommentTok{\#   group\_by(lockdown\_duration) \%\textgreater{}\%}
\CommentTok{\#   summarise(}
\CommentTok{\#     Mean\_NoIntervention = mean(model\_tot\_econ\_loss),}
\CommentTok{\#     Mean\_Traditional = mean(model\_trad\_tot\_econ\_loss),}
\CommentTok{\#     Mean\_ML = mean(model\_ml\_tot\_econ\_loss),}
\CommentTok{\#     Mean\_ML\_Improvement = mean(ml\_improvement),}
\CommentTok{\#     Mean\_Trad\_Improvement = mean(trad\_improvement),}
\CommentTok{\#     Mean\_ML\_Pct = mean(ml\_pct\_improvement),}
\CommentTok{\#     .groups = \textquotesingle{}drop\textquotesingle{}}
\CommentTok{\#   )}
\CommentTok{\# }
\CommentTok{\# print(summary\_stats)}
\end{Highlighting}
\end{Shaded}

\begin{Shaded}
\begin{Highlighting}[]
\CommentTok{\# lockdown\_duration\_vals = seq(10,55)}
\CommentTok{\# total\_duration\_vals = seq(300,800)}
\CommentTok{\# }
\CommentTok{\# nsim = length(lockdown\_duration\_vals) * length(total\_duration\_vals)}
\CommentTok{\# }
\CommentTok{\# col\_names = c("lockdown\_duration",}
\CommentTok{\#     "total\_duration",}
\CommentTok{\#     "model\_tot\_econ\_loss",}
\CommentTok{\#     "model\_trad\_tot\_econ\_loss",}
\CommentTok{\#     "model\_ml\_tot\_econ\_loss")}
\CommentTok{\# }
\CommentTok{\# sim\_matrix = matrix(data=NA, nrow=nsim , ncol=5)}
\CommentTok{\# colnames(sim\_matrix) = col\_names}
\CommentTok{\# }
\CommentTok{\# row\_idx = 1}
\CommentTok{\# for (l\_duration in lockdown\_duration\_vals)\{}
\CommentTok{\#   for (t\_duration in total\_duration\_vals)\{}
\CommentTok{\#     res = simulation\_ml\_vs\_diim(q0, A\_star,c\_star,x,lockdown\_duration=l\_duration, total\_duration=t\_duration)}
\CommentTok{\#     res = matrix(unlist(res), ncol=5)}
\CommentTok{\#     sim\_matrix[row\_idx,] = res}
\CommentTok{\#     }
\CommentTok{\#     row\_idx = row\_idx + 1}
\CommentTok{\#   \}}
\CommentTok{\# \}}
\end{Highlighting}
\end{Shaded}




\end{document}
